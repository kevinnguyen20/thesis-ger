\chapter{Einleitung}
Länge: ca. 1 - 5. Seiten\\\\

\noindent Aufbau:
\begin{itemize}
	\item Motivation der Arbeit /Problem
	\item Ansatz der Lösung / Ziele
	\item Struktur der Arbeit / Vorgehen
\end{itemize}
Die Einleitung dient dazu, beim Leser Interesse für das Thema der Arbeit zu wecken, das behandelte Problem aufzuzeigen und den zu seiner Lösung eingeschlagenen Weg zu beschreiben. In diesem Kapitel wird die mit dem Betreuer/Professor besprochene Aufgabenstellung herausgearbeitet und für einen potentiellen Leser "spannend" dargestellt.\\\\

\noindent Motivation:\\
In der Motivation wird dargestellt, wieso es notwendig ist, sich mit dem in der Arbeit identifizierten und behandelten Problem zu beschäftigen. Zur Entwicklung der Motivation kann eine dem Leser bekannte Problematik aufgegriffen und dann die Problemstellung hieraus abgeleitet werden. Der Betreuer unterstützt die Entwicklung der Motivation, indem er bei der Einordnung der Arbeit in ein größeres Problemgebiet hilft.\\\\

\noindent Häufige Fehler:
\begin{itemize}
	\item Zu allgemeine Motivation, Problemstellung und -abgrenzung. Die Problemstellung beginnt mit der Einordnung in ein thematisches Umfeld und enthält sowohl die in der Arbeit angegangenen Problempunkte, als auch weitere, nicht behandelte Problempunkte. Eine Negativabgrenzung verhindert, dass beim Leser später nicht erfüllte Erwartungen geweckt werden.
Der Betreuer unterstützt die Eingrenzung der Problemstellung, indem er Hinweise auf abzugrenzende Punkte bzw. auszuschließende Punkte im Rahmen der Negativabgrenzung gibt.
\noindent Häufige Fehler:
	\item Keine klare Problemstellung und -abgrenzung
	\item Fehlen der Negativabgrenzung
\end{itemize}

\noindent Ziel der Arbeit:\\
Mit dem Ziel der Arbeit wird der angestrebte Lösungsumfang festgelegt. An diesem Ziel wird die Arbeit gemessen.
Der Betreuer sorgt dafür, dass das Ziel der Arbeit realisierbar und im Rahmen einer Diplomarbeit lösbar ist.\\\\
\noindent Häufige Fehler:
\begin{itemize}
	\item Kein klares Ziel
	\item Zu viele Ziele
\end{itemize}


\noindent Vorgehen:\\
Nachdem mit Problemstellung und Ziel gewissermaßen Anfangs- und Endpunkt der Arbeit beschrieben sind, wird hier der zur Erreichung des Ziels eingeschlagene Weg vorgestellt. Dazu werden typischerweise die folgenden Kapitel und ihr Beitrag zur Erreichung des Ziels der Arbeit kurz beschrieben. Die folgenden Kapitel sind ein {\em möglicher} Aufbau, Abweichungen können durchaus notwendig sein. Zur Darstellung des Vorgehens kann eine grafische Darstellung sinnvoll sein, bei der die einzelnen Lösungsschritte und ihr Zusammenhang dargestellt werden.


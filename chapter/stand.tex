\chapter{Grundlagen und Stand der Forschung}
Länge: ca. 5-10 Seiten\\\\

\noindent In diesem Kapitel werden für die weitere Arbeit wichtige Begriffe eingeführt. Dabei ist darauf zu achten, nur solche Inhalte in das Grundlagenkapitel aufzunehmen, die später auch verwendet werden (Problembezogenheit). Ebenso ist auf eine ausreichend Tiefe und vollständige Darstellung der Grundlagen zu achten. Des Weiteren müssen verwandte schon vorhandene Arbeiten aus dem bearbeiteten Forschungsgebiet hier aufgeführt werden. Diese verwandten Arbeiten sind kurz zu analysieren. Wo immer möglich sind Referenzen auf vorhandene Literatur einzusetzen. d.h. nur wenn von der Literatur abweichende Definitionen und Konzepte verwendet werden, ist eine ausführliche Darstellung von Definitionen und Konzepten begründet. Die Darstellung von Definitionen und Konzepten muss unbedingt homogen und widerspruchsfrei dargestellt werden. Keinesfalls dürfen beispielsweise mehrere Definitionen des gleichen Begriffes nebeneinander gestellt werden, ohne dass eine begründete Entscheidung für die letztlich in der Arbeit verwendete Definition getroffen wird. Ein weiterer Punkt ist die Vollständigkeit der Grundlagen. So sollten alle möglichen Merkmalskombinationen abgedeckt werden, was beispielsweise mit Hilfe einer Tabelle geschehen kann.

\noindent Der Betreuer gibt Hinweise auf relevante Literatur.\\\\

\noindent Häufige Fehler sind:
\begin{itemize}
	\item Zuviel Grundlagen ohne Notwendigkeit für den Problemlösungsprozess
	\item	Bloßes Nebeneinanderstellen von Definitionen ohne Auswahl einer für die Arbeit verbindlichen Definition
	\item Bloßes Nebeneinanderstellen von Definitionen ohne logischen Fluss
	\item	Unstrukturiertes Aneinanderreihen von Literaturzitaten ohne Beitrag zum Problemlösungsprozess
\end{itemize}

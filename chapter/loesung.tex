\chapter{Lösungskonzept}
Länge: ca. 5 - 15 Seiten\\\\
Im Lösungskonzept wird auf konzeptueller Ebene der Weg zur Lösung der identifizierten Probleme beschrieben. Ausgangspunkt sind die Erkenntnisse der vorangegangenen Problemanalyse. Wichtig ist hierbei die Herausstellung des erzielten Neuigkeits- und Innovationswertes im Bezug auf den bisherigen Stand der Technik/Wissenschaft. Grundlage hierfür ist ebenfalls die im vorangegangenen Kapitel durchgeführte Problemanalyse. Im Lösungskapitel werden noch keine umsetzungsspezifischen Details angeführt, dies ist Aufgabe des folgenden Kapitels. Eine typische Gliederung für die Darstellung des Lösungskonzepts ist das Aufgreifen der im vorangegangenen Kapitel identifizierten Problembereiche. Der Betreuer berät bei der Darstellung des Lösungskonzepts.\\\\

\noindent Häufige Fehler:
\begin{itemize}
	\item Lösungskonzept passt nicht zum Ziel
	\item Lösungskonzept enthält Bestandteile der Umsetzung
\end{itemize}

\noindent Kapitelzusammenfassung am Ende:\\
Eine Zusammenfassung erleichtert es dem Leser, die erarbeitete Lösung zu erfassen.

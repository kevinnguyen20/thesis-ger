\documentclass[12pt,a4paper]{report}
%\documentclass[12pt,a4paper,onepage]{scrbook}

\makeatletter

\def\author#1{\gdef\insertauthor{#1}\gdef\@author{#1}\hypersetup{pdfauthor={#1}}}
\def\title#1{\gdef\inserttitle{#1}\gdef\@title{#1}\hypersetup{pdftitle={#1}}}
\def\keywords#1{\gdef\insertkeywords{#1}\hypersetup{pdfkeywords={#1}}}
\def\subject#1{\gdef\insertsubject{#1}\hypersetup{pdfsubject={#1}}}

\makeatother

\RequirePackage[final,breaklinks,bookmarks]{hyperref}

\hypersetup{%
    pdffitwindow=true,
    pdfmenubar=true,
    frenchlinks=false,
    colorlinks=false,
    bookmarksopen=true,
    bookmarksnumbered=true,
    pagebackref=true,
    pdfpagelabels=true,
    pdfstartview=FitH,
    pdfcreator  = {\LaTeX\ with package \flqq hyperref\frqq},
    pdfproducer = {pdfeTeX-0.\the\pdftexversion\pdftexrevision}
}


%############################################################################
%########################### Change This ####################################
%############################################################################
%replace with German "Bachelor Thesis", "Master Thesis" or "Diplomarbeit" or
%replace with English "Bachelor's Thesis", "Master's Thesis"  
\subject{Master/Bachelor Thesis}
%put your title here
\title{Title of my Thesis}
%your name
\author{Manfred Mustermann}
% your matrikelnummer
\newcommand{\trmatrikelnummer}{12345}
%supervisor
\newcommand{\trbetreuerA}{Dipl.-Inf. Aubrey-Derrick Schmidt}
%\newcommand{\trbetreuerA}{Dipl.-Inf. Christian Scheel}
%reviewer 1
\newcommand{\trguta}{Prof. Dr. Dr. h.c. Sahin Albayrak}
%reviewer two
\newcommand{\trgutb}{Prof. Dr. Odej Kao}
\date{\today}
%Auswahl passender Keywords, dass sind nur Beispiele
\keywords{Multi-Agent Systems, Machine Learning}

%############################################################################
%############################################################################

\usepackage[utf8]{inputenc}
\usepackage[T1]{fontenc}
\newcommand{\bmpstego}{stego}
\newcommand{\changefont}[3]{
\fontfamily{#1} \fontseries{#2} \fontshape{#3} \selectfont}

% Sprachen:
\usepackage[ngerman]{babel} % Silbentrennung Deutsch neue Rechtschreibung
\selectlanguage{ngerman}

\sloppy
%\usepackage{makeidx}
%\makeglossary
\makeindex

\usepackage{sty/abbreviations}

\usepackage{amsmath, marvosym} % Mathematik
\usepackage{times, url, geometry, amssymb, graphicx, booktabs}
\usepackage{fancyhdr} %Kopf- und Fußzeilen
\usepackage[hyphenbreaks]{breakurl}
\usepackage{color} % Farben

\usepackage{subfigure} % mehrere Abbildungen nebeneinander/übereinander
\usepackage{latexsym}

\geometry{a4paper,body={5.8in,9in}}
\setlength{\headheight}{15pt}

\usepackage{setspace} % 1,5 Zeilenabstand
\onehalfspacing
\setcounter{secnumdepth}{4}
\setcounter{tocdepth}{3} 

% Hurenkinder- und Schusterjungenregelung
\clubpenalty = 10000 % schliesst Schusterjungen aus
\widowpenalty = 10000 % schliesst Hurenkinder aus

% aller Bilder werden im Unterverzeichnis figures gesucht:
\graphicspath{{figures/}}

% Headers:
%\pagestyle{headings}
\pagestyle{fancy}
\pagestyle{headings}

% Literaturverzeichnis
\usepackage{bibgerm}
%\usepackage{natbib}
\bibliographystyle{plaindin} % Literaturangaben nach Auftreten sortieren %{gerplain}

\usepackage{listings} % für Formatierung in Quelltexten
\definecolor{grau}{gray}{0.25}
\lstset{
	extendedchars=true,
	basicstyle=\scriptsize\ttfamily,
	%basicstyle=\tiny\ttfamily,
	tabsize=2,
	keywordstyle=\textbf,
	commentstyle=\color{grau},
	stringstyle=\textit,
	numbers=left,
	numberstyle=\tiny,
	% für schönen Zeilenumbruch
	breakautoindent  = true,
	breakindent      = 2em,
	breaklines       = true,
	postbreak        = ,
	%prebreak         = \raisebox{-.8ex}[0ex][0ex]{\ensuremath{\lrcorner}},
	prebreak         = \raisebox{-.8ex}[0ex][0ex]{\Righttorque},
}

